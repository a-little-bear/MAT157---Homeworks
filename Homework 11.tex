\documentclass{homework}
\author{Joseph Siu}
\class{MAT157: Analysis I}
\date{\today}
\title{Homework 11}

\newcommand{\Set}[1]{\{#1\}}
\newcommand{\T}[1]{\text{#1}}
\newcommand{\Al}[3]{#1 &=#2 &\text{#3}&&\\}

% Symbols
\newcommand*{\eg}{\leavevmode\unskip , e. g., \ignorespaces} % for example
\newcommand*{\ie}{\leavevmode\unskip, i. e., \ignorespaces} % that is
\newcommand{\nil}{\varnothing}
\AtBeginDocument{\def\O{\cal{O}}} % Big Oh
\AtBeginDocument{\def\C{\bb{C}}} % Complex
\newcommand{\R}{\bb{R}} % Reals
\newcommand{\Q}{\bb{Q}} % Rationals
\newcommand{\Z}{\bb{Z}} % Integers
\newcommand{\N}{\bb{N}} % Naturals
\renewcommand{\P}{\bb{P}} % Primes
\newcommand{\Pset}[1]{\mathcal{P}(#1)} %power set
\newcommand{\Relate}[2]{#1\mathcal{R}#2} %relation
\newcommand{\relate}{\mathcal{R}}
\newcommand{\F}{\bb{F}} 
\newcommand{\GF}[1][2]{\bb{F}_{#1}} 
\newcommand{\modulo}[1][n]{\Z/#1\Z} 
\newcommand{\ra}{\rightarrow}
\newcommand{\Ra}{\Rightarrow}
\newcommand{\?}{\stackrel{?}{=}}
\newcommand{\is}{\equiv}
\newcommand{\al}{\alpha}
\newcommand{\ep}{\varepsilon}
\renewcommand{\phi}{\varphi}
\newcommand{\p}{\partial}
\newcommand{\injective}{\hookrightarrow}
\newcommand{\surjective}{\twoheadrightarrow}
\newcommand{\bijective}{\hookrightarrow\mathrel{\mspace{-15mu}}\rightarrow}
\newcommand{\derivative}[2][x]{\frac{\D #2}{\D #1}}
\newcommand{\ceil}[1]{\left\lceil#1\right\rceil}
\newcommand{\floor}[1]{\left\lfloor#1\right\rfloor}
\newcommand{\near}[1]{\left\lfloor#1\right\rceil}
\newcommand{\arr}[1]{\left\langle#1\right\rangle}
\newcommand{\paren}[1]{\left(#1\right)} %pair / ()
\newcommand{\brk}[1]{\left[#1\right]} %[]
\newcommand{\abs}[1]{\left|#1\right|}
\newcommand{\curl}[1]{\left\{#1\right\}} %set {}
\newcommand{\func}[3]{#1: #2 \rightarrow #3}


\theoremstyle{definition}
\newtheorem*{claim}{Claim}
\newtheorem*{definition}{Definition}
\newtheorem*{theorem}{Theorem}
\newtheorem*{lemma}{Lemma}


\begin{document} \maketitle

\section*{Exercise 1}

Let $f(x)$ be a function defined near 0 and $\displaystyle\lim_{x\to0}f(x)=0$.

\question[1] Prove that $\textup{if~}g(x)=o(\mathcal{O}(f(x))),\mathrm{~then~}g(x)=o(f(x))$. 

\question[2] Prove that $\text{if~}g(x)=\mathcal{O}(o((f(x))),\text{~then~}g(x)=o(f(x))$.

\newpage
\section*{Exercise 2}
Let the angle $\angle AOB=x$. Find $n\in\N$ so that the following quantity $g(x)$ satisfies that $g(x)=\cal{O}(x^n)$ and $x^n=\cal{O}(g(x))$.

\fig{image.png}{Exercise 2}{}{}

\question[1] The chord length $|AB|$.

\begin{definition}
    We denote $f(x)\sim g(x)$ when $x\to0$ if $\lim_{x\to0}\frac{f(x)}{g(x)}=1$.
\end{definition}

\begin{lemma}
    $2\sin(\frac{x}2)\sim x$ when $x\to0$.

    \begin{proof}
        As proven, we have $\lim_{x\to0}\frac{\sin(x)}{x}=1$, thus \begin{align*}
            \lim_{x\to0}\frac{2\cdot\sin(\frac{x}2)}{x}&=\lim_{x\to0}2\cdot\frac{\sin(\frac{x}2)}{\frac{x}2}\cdot\frac{\frac{x}2}{x}\\
            &=\lim_{x\to0}2\cdot\frac{\sin(\frac{x}2)}{\frac{x}2}\cdot\frac12\\
            &=2\cdot\frac12\\
            &=1
        \end{align*}
        by definition showing $2\sin(\frac{x}2)\sim x$ when $x\to0$.
    \end{proof}
\end{lemma}

\begin{lemma}
    If $f(x),g(x)$ are bounded and continuous functions when $x\in[-1,1]$, then \[\limsup_{x\to0}|f(x)g(x)|\leq\limsup_{x\to0}|f(x)|\cdot\limsup_{x\to0}|g(x)|\]

    \begin{proof}
        By Bolzano-Weierstrass Theorem, there must exist a convergent seqeunce $(x_n y_n)$ where $\forall n\in\N, x_n\in |f([-1,1])|, y_n\in |g([-1,1])|$ s.t. $\lim_{n\to\infty}(x_n y_n)=\limsup_{x\to0}|f(x)g(x)|\in\R$ (by lecture [-1,1] is closed and $f,g$ are continuous mean the supremum $|f(x)g(x)|$ must be achieved). 
        
        Thus, $0\leq\lim_{n\ra\infty}x_n\leq\limsup_{n\ra\infty}x_n\leq\limsup_{x\ra0}|f(x)|$ and $0\leq\lim_{n\ra\infty}y_n\leq\limsup_{n\ra\infty}y_n\leq\limsup_{x\ra0}|g(x)|$ give $$\limsup_{x\ra0}|f(x)g(x)|=\lim_{n\ra\infty}(x_n y_n)\leq\limsup_{n\ra\infty}x_n\cdot\limsup_{n\ra\infty}y_n\leq \limsup_{x\to0}|f(x)|\cdot\limsup_{x\to0}|g(x)|,$$ as required. 
    \end{proof}
\end{lemma}

\begin{lemma}
    Assume $f,g$ are continuous functions. If $x\to0, f(x)\sim g(x), f(x), g(x)\neq0$, then when $x\to0$, $f(x)=\cal{O}(x^n)\iff g(x)=\cal{O}(x^n)$ for some fixed $n\in\N$.

    \begin{proof}
        Since $f(x)\sim g(x)\iff g(x)\sim f(x)$, w.l.o.g. we just need to show $f(x)=\cal{O}(x^n)\implies g(x)=\cal{O}(x^n)$. By definition, we have \[\lim_{x\to0}\sup|\frac{f(x)}{x^n}|\leq M, M\geq0,\] and $f$ is bounded since $x^n$ is bounded on $[-1,1]$, which also implies $g$ is bounded as $f$ is bounded and $f(x)\sim g(x)$ when $x\to0$. Then, by the previous lemma \begin{align*}
            \lim_{x\to0}\sup|\frac{f(x)}{x^n}|&=\lim_{x\to0}\sup|\frac{f(x)}{x^n}|\cdot\lim_{x\to0}\sup|1|\\
            &=\lim_{x\to0}\sup|\frac{f(x)}{x^n}|\cdot\lim_{x\to0}\sup|\frac{g(x)}{f(x)}|\\
            &\geq \lim_{x\to0}\sup|\frac{f(x)}{x^n}\cdot\frac{g(x)}{f(x)}|\\
            &=\lim_{x\to0}\sup|\frac{g(x)}{x^n}|\\
        \end{align*}
        showing \[\lim_{x\to0}\sup|\frac{g(x)}{x^n}|\leq\lim_{x\to0}\sup|\frac{f(x)}{x^n}|\leq M\], which means $g(x)=\cal{O}(x^n)$, as required.
    \end{proof}
\end{lemma}

\begin{lemma}
    When $x\to0$, $Cx^n=\cal{O}(x^n)$ for all $C\in\R$, for all $n\in\N$. 

    \begin{proof}
        $\limsup_{x\to0}|\frac{Cx^n}{x^n}|=|C|\leq |C|$ where $|C|\geq0$, by definition showing $Cx^n=\cal{O}(x^n)$ for any $n\in\N$ and any $C\in\R$.
    \end{proof}
\end{lemma}

\begin{proof}
    Assume $x\to0$. Denote the radius $R=AO=CO=BO>0$, then by lemmas and formula of triangle we have:

    \(g(x)=|AB|=2R\sin(\frac{x}2)\sim Rx\), since $R\in\R$, choose $n=1$ we have \(g(x)\sim Rx=\cal{O}(x)\) and $x=\cal{O}(Rx)\implies x=\cal{O}(g(x))$ since \begin{align*}
        \limsup_{x\to0}\frac{x}{Rx}&=\limsup_{x\to0}\frac{x}{Rx}\cdot\limsup_{x\to0}1\\
        &=\limsup_{x\to0}\frac{x}{Rx}\cdot\limsup_{x\to0}\frac{Rx}{g(x)}\\
        &\geq\limsup_{x\to0}\frac{x}{Rx}\cdot\frac{Rx}{g(x)}\\
        &=\limsup_{x\to0}\frac{x}{g(x)}\\
    \end{align*} which gives $x=\cal{O}(g(x))$.
\end{proof}

\question[2] The arch height $|CD|$.

\begin{lemma}
    $1-\cos(\frac{x}2)\sim \frac{x^2}8$ when $x\to0$.

    \begin{proof}
        By l'hopital's rule we have \begin{align*}
            \lim_{x\to0}\frac{1-\cos(\frac{x}2)}{\frac{x^2}8}&=\lim_{x\to0}\frac{8-8\cos(\frac{x}2)}{x^2}\\
            &=\lim_{x\to0}\frac{4\sin(\frac{x}2)}{2x}\\
            &=\lim_{x\to0}\cos(\frac{x}2)\\
            &=1
        \end{align*}
    \end{proof}
\end{lemma}

\begin{proof}
    Assume $n\to0$ and D is the mid point of line $AB$. Then, $g(x)=|CD|=R-R\cos(\frac{x}2)\sim \frac{Rx^2}{8},$ choose $n=2$, then similar to E2Q1 we have $g(x)=\cal{O}(x^2)$ and $x^2=\cal{O}(g(x))$. 
\end{proof}

\question[3] Area of the sector $AOB$.

\begin{proof}
    Denote A as the area of the sector $AOB$. Then, $g(x)=A=\frac12R^2x$, choose $n=1$, similar to E2Q1 we have $g(x)=\cal{O}(x)$ and $x=\cal{O}(g(x))$. 
\end{proof}

\question[4] Area of the triangle $\triangle ACB$.

\begin{lemma}
    $\sin(\frac{x}2)(1-\cos(\frac{x}2))\sim \frac{x^3}{16}$ when $x\to0$.

    \begin{proof}
        By the previous lemmas we have \begin{align*}
            \lim_{x\to0}\frac{\sin(\frac{x}2)(1-\cos(\frac{x}2))}{\frac{x^3}{16}}&=\lim_{x\to0}\frac{\sin(\frac{x}2)}{\frac{x}{2}}\cdot\lim_{x\to0}\frac{1-\cos(\frac{x}2)}{\frac{x^2}{8}}\\
            &=1\cdot1\\
            &=1
        \end{align*}
    \end{proof}
\end{lemma}

\begin{proof}
    Assume $x\to0$ and $D$ is the mid point of line $AB$. Denote $A$ as the function returning the area of a triangle. Then, $A(\triangle ACB)=A(\triangle ACO)+A(\triangle BCO)-A(\triangle ABO)$. That is, $g(x)=A(\triangle ACB)=2\cdot\frac12R^2\sin(\frac{x}2)-\frac12R^2\sin(x)=R^2\sin(\frac{x}2)-R^2\sin(\frac{x}2)\cos(\frac{x}{2})=R^2\sin(\frac{x}2)(1-\cos(\frac{x}{2}))\sim \frac{R^2x^3}{16}$, choose $n=3$, then similar to E2Q1 we have $g(x)=\cal{O}(x^3)$ and $x^3=\cal{O}(g(x))$.
\end{proof}


\newpage
\section*{Exercise 3}

Consider the function $f:\R\to\R$ given by \[f(x)=e^{x^2+\frac{\sin(x)}{1+x^2}}\]

\question[1] Compute the approximation of the value $f(1.001)$ by using linear approximation.

\begin{align*}
    f'(x)&=(e^{x^2+\frac{\sin(x)}{1+x^2}})'\\
    &=(x^2+\frac{\sin(x)}{1+x^2})'e^{x^2+\frac{\sin(x)}{1+x^2}}\\
    &=((x^2)'+(\frac{\sin(x)}{1+x^2})')e^{x^2+\frac{\sin(x)}{1+x^2}}\\
    &=(2x+(\frac{\sin(x)}{1+x^2})')e^{x^2+\frac{\sin(x)}{1+x^2}}\\
    &=(2x+\frac{\cos(x)(1+x^2)-\sin(x)(2x)}{(1+x^2)^2})e^{x^2+\frac{\sin(x)}{1+x^2}}\\
\end{align*}

Using the formula $f(x+\Delta x)-f(x)=f'(x)\Delta x+o(\Delta x)$, where $x=1, \Delta x=0.001$, isolate $f(x+\Delta x)$ we have

\begin{align*}
    f(x+\Delta x) &= f(x) + f'(x)\Delta x + o(\Delta x)\\
    &\approx f(1) + 0.001\cdot f'(1)\\
    &\approx ...
\end{align*}

\question[2] Now suppose that you need to ensure the tolerance of error is less or equal to the scale of $10^{-17}$. Normally speaking, how many terms in the Taylor expansion approximation do you need, given that in our scenario $\Delta x=0.001?$

\begin{proof}
    Since $\Delta x$ is the differences of $e^x$ and the Taylor expansion approximation of $e^x$ $\Delta x=e^x$.
\end{proof}

\end{document}
\documentclass{homework}
\author{Joseph Siu}
\class{MAT157: Analysis I}
\date{\today}
\title{Homework 11}

\newcommand{\Set}[1]{\{#1\}}
\newcommand{\T}[1]{\text{#1}}
\newcommand{\Al}[3]{#1 &=#2 &\text{#3}&&\\}

% Symbols
\newcommand*{\eg}{\leavevmode\unskip , e. g., \ignorespaces} % for example
\newcommand*{\ie}{\leavevmode\unskip, i. e., \ignorespaces} % that is
\newcommand{\nil}{\varnothing}
\AtBeginDocument{\def\O{\cal{O}}} % Big Oh
\AtBeginDocument{\def\C{\bb{C}}} % Complex
\newcommand{\R}{\bb{R}} % Reals
\newcommand{\Q}{\bb{Q}} % Rationals
\newcommand{\Z}{\bb{Z}} % Integers
\newcommand{\N}{\bb{N}} % Naturals
\renewcommand{\P}{\bb{P}} % Primes
\newcommand{\Pset}[1]{\mathcal{P}(#1)} %power set
\newcommand{\Relate}[2]{#1\mathcal{R}#2} %relation
\newcommand{\relate}{\mathcal{R}}
\newcommand{\F}{\bb{F}} 
\newcommand{\GF}[1][2]{\bb{F}_{#1}} 
\newcommand{\modulo}[1][n]{\Z/#1\Z} 
\newcommand{\ra}{\rightarrow}
\newcommand{\Ra}{\Rightarrow}
\newcommand{\?}{\stackrel{?}{=}}
\newcommand{\is}{\equiv}
\newcommand{\al}{\alpha}
\newcommand{\ep}{\varepsilon}
\renewcommand{\phi}{\varphi}
\newcommand{\p}{\partial}
\newcommand{\injective}{\hookrightarrow}
\newcommand{\surjective}{\twoheadrightarrow}
\newcommand{\bijective}{\hookrightarrow\mathrel{\mspace{-15mu}}\rightarrow}
\newcommand{\derivative}[2][x]{\frac{\D #2}{\D #1}}
\newcommand{\ceil}[1]{\left\lceil#1\right\rceil}
\newcommand{\floor}[1]{\left\lfloor#1\right\rfloor}
\newcommand{\near}[1]{\left\lfloor#1\right\rceil}
\newcommand{\arr}[1]{\left\langle#1\right\rangle}
\newcommand{\paren}[1]{\left(#1\right)} %pair / ()
\newcommand{\brk}[1]{\left[#1\right]} %[]
\newcommand{\abs}[1]{\left|#1\right|}
\newcommand{\curl}[1]{\left\{#1\right\}} %set {}
\newcommand{\func}[3]{#1: #2 \rightarrow #3}


\theoremstyle{definition}
\newtheorem*{claim}{Claim}
\newtheorem*{definition}{Definition}
\newtheorem*{theorem}{Theorem}
\newtheorem*{lemma}{Lemma}


\begin{document} \maketitle

\section*{Exercise 1}

Let $f(x)$ be a function defined near 0 and $\displaystyle\lim_{x\to0}f(x)=0$.

\question[1] Prove that $\textup{if~}g(x)=o(\mathcal{O}(f(x))),\mathrm{~then~}g(x)=o(f(x))$. 

\question[2] Prove that $\text{if~}g(x)=\mathcal{O}(o((f(x))),\text{~then~}g(x)=o(f(x))$.

\newpage
\section*{Exercise 2}
Let the angle $\angle AOB=x$. Find $n\in\N$ so that the following quantity $g(x)$ satisfies that $g(x)=\cal{O}(x^n)$ and $x^n=\cal{O}(g(x))$.

\fig{image.png}{Exercise 2}{}{}

\question[1] The chord length $|AB|$.

\question[2] The arch height $|CD|$.

\question[3] Area of the sector $AOB$.

\question[4] Area of the triangle $\triangle ACB$.



\newpage
\section*{Exercise 3}

Consider the function $f:\R\to\R$ given by \[f(x)=e^{x^2+\frac{\sin(x)}{1+x^2}}\]

\question[1] Compute the approximation of the value $f(1.001)$ by using linear approximation.

\begin{align*}
    f'(x)&=(e^{x^2+\frac{\sin(x)}{1+x^2}})'\\
    &=(x^2+\frac{\sin(x)}{1+x^2})'e^{x^2+\frac{\sin(x)}{1+x^2}}\\
    &=((x^2)'+(\frac{\sin(x)}{1+x^2})')e^{x^2+\frac{\sin(x)}{1+x^2}}\\
    &=(2x+(\frac{\sin(x)}{1+x^2})')e^{x^2+\frac{\sin(x)}{1+x^2}}\\
    &=(2x+\frac{\cos(x)(1+x^2)-\sin(x)(2x)}{(1+x^2)^2})e^{x^2+\frac{\sin(x)}{1+x^2}}\\
\end{align*}

Using the formula $f(x+\Delta x)-f(x)=f'(x)\Delta x+o(\Delta x)$, where $x=1, \Delta x=0.001$, isolate $f(x+\Delta x)$ we have

\begin{align*}
    f(x+\Delta x) &= f(x) + f'(x)\Delta x + o(\Delta x)\\
    &\approx f(1) + 0.001\cdot f'(1)\\
    &\approx ...
\end{align*}

\question[2] Now suppose that you need to ensure the tolerance of error is less or equal to the scale of $10^{-17}$. Normally speaking, how many terms in the Taylor expansion approximation do you need, given that in our scenario $\Delta x=0.001?$

\end{document}
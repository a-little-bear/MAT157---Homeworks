\documentclass{homework}
\author{Joseph Siu}
\class{MAT157: Analysis I}
\date{\today}
\title{Homework 9}

\newcommand{\Set}[1]{\{#1\}}
\newcommand{\T}[1]{\text{#1}}
\newcommand{\Al}[3]{#1 &=#2 &\text{#3}&&\\}

% Symbols
\newcommand*{\eg}{\leavevmode\unskip , e. g., \ignorespaces} % for example
\newcommand*{\ie}{\leavevmode\unskip, i. e., \ignorespaces} % that is
\newcommand{\nil}{\varnothing}
\AtBeginDocument{\def\O{\cal{O}}} % Big Oh
\AtBeginDocument{\def\C{\bb{C}}} % Complex
\newcommand{\R}{\bb{R}} % Reals
\newcommand{\Q}{\bb{Q}} % Rationals
\newcommand{\Z}{\bb{Z}} % Integers
\newcommand{\N}{\bb{N}} % Naturals
\renewcommand{\P}{\bb{P}} % Primes
\newcommand{\Pset}[1]{\mathcal{P}(#1)} %power set
\newcommand{\Relate}[2]{#1\mathcal{R}#2} %relation
\newcommand{\relate}{\mathcal{R}}
\newcommand{\F}{\bb{F}} 
\newcommand{\GF}[1][2]{\bb{F}_{#1}} 
\newcommand{\modulo}[1][n]{\Z/#1\Z} 
\newcommand{\ra}{\rightarrow}
\newcommand{\Ra}{\Rightarrow}
\newcommand{\?}{\stackrel{?}{=}}
\newcommand{\is}{\equiv}
\newcommand{\al}{\alpha}
\newcommand{\ep}{\varepsilon}
\renewcommand{\phi}{\varphi}
\newcommand{\p}{\partial}
\newcommand{\injective}{\hookrightarrow}
\newcommand{\surjective}{\twoheadrightarrow}
\newcommand{\bijective}{\hookrightarrow\mathrel{\mspace{-15mu}}\rightarrow}
\newcommand{\derivative}[2][x]{\frac{\D #2}{\D #1}}
\newcommand{\ceil}[1]{\left\lceil#1\right\rceil}
\newcommand{\floor}[1]{\left\lfloor#1\right\rfloor}
\newcommand{\near}[1]{\left\lfloor#1\right\rceil}
\newcommand{\arr}[1]{\left\langle#1\right\rangle}
\newcommand{\paren}[1]{\left(#1\right)} %pair / ()
\newcommand{\brk}[1]{\left[#1\right]} %[]
\newcommand{\abs}[1]{\left|#1\right|}
\newcommand{\curl}[1]{\left\{#1\right\}} %set {}
\newcommand{\func}[3]{#1: #2 \rightarrow #3}


\theoremstyle{definition}
\newtheorem*{claim}{Claim}
\newtheorem*{definition}{Definition}
\newtheorem*{theorem}{Theorem}
\newtheorem*{lemma}{Lemma}


\begin{document} \maketitle

\section*{Exercise 1}

Compute the following limits.

\question[1] 

\begin{claim}
\[\lim_{x\to1}\frac{x+x^2+x^3+\ldots+x^n-n}{x-1}=\frac{n(n+1)}{2}\]
\end{claim}

\begin{proof}
    Replace $x$ with $1+h$ where $h\to0$ as $x\to1$. Then by binomial theorem we have \begin{align*}
        \lim_{x\to1}\frac{x+x^2+x^3+\ldots+x^n-n}{x-1}&=\lim_{h\to0}\frac{(1+h)+(1+h)^2+\ldots+(1+h)^n-n}{h}\\
        &=1+2+\ldots+n\\
        &=\frac{n(n+1)}{2}
    \end{align*} 
    The constant terms of the binomial terms are cancelled with the $-n$, then we are able to factor out $h$ from all numerator terms, cancel it with the denominator's $h$ we have many terms with $h$ left and $1+2+\ldots+n$, however as $h$ approaches $0$, all terms with $h$ will approach $0$ and we are left with $1+2+\ldots+n$, thus giving us $\frac{n(n+1)}{2}$ as needed.
\end{proof}

\question[2] 

\begin{claim}
    \[\lim_{x\ra1}\left( \frac{m}{1-x^m}-\frac{n}{1-x^n} \right)=\frac{m-n}{2}\]
\end{claim}

\begin{proof}
    Using the identity $1-x^a=(1-x)(1+x+\cdots+x^{a-1})$ we have:
    \begin{align*}
        \lim_{x\ra1}\left( \frac{m}{1-x^m}-\frac{n}{1-x^n} \right) &= \lim_{x\ra1}\left( \frac{m(1-x^n)-n(1-x^m)}{(1-x^m)(1-x^n)} \right)\\
        &=\lim_{x\ra1}\left( \frac{m(1-x)(1+x+\cdots+x^{n-1})-n(1-x)(1+x+\cdots+x^{m-1})}{(1-x)(1+x+\cdots+x^{m-1})(1-x)(1+x+\cdots+x^{n-1})} \right)\\
        &=\lim_{x\ra1}\left( \frac{m(1-x)(1+x+\cdots+x^{n-1})-n(1-x)(1+x+\cdots+x^{m-1})}{(1-x)m(1-x)n} \right)\text{ since $x\ra1$}\\
        &=\frac1{mn}\lim_{x\ra1}\left( \frac{m(1+x+\cdots+x^{n-1})-n(1+x+\cdots+x^{m-1})}{(1-x)} \right)\\
        &=\frac1{mn}\lim_{x\ra1}\left( \frac{m(x+\cdots+x^{n-1})+m+mn-mn-n-n(x+\cdots+x^{m-1})}{(1-x)} \right)\\
        &=\frac1{mn}\lim_{x\ra1}\left( \frac{m(x+\cdots+x^{n-1}-(n-1))-n(x+\cdots+x^{m-1}-(m-1))}{(1-x)} \right)\\
        &=-\frac1{mn}\lim_{x\ra1}\left( m\frac{x+\cdots+x^{n-1}-(n-1)}{x-1}-n\frac{x+\cdots+x^{m-1}-(m-1)}{x-1} \right)\\
        &=-\frac1{mn}\left( \frac{m(n-1)n}{2} - \frac{n(m-1)m}{2}\right) \text{ by E1Q1}\\
        &=\frac{m-n}{2}
    \end{align*}
\end{proof}



\question[3] 

\begin{claim}
    \[\lim_{x\ra4}\frac{\sqrt{1+2x}-3}{\sqrt{x}-2}=\frac43\]
\end{claim}

\begin{proof}
    \begin{align*}
        \lim_{x\ra4}\frac{\sqrt{1+2x}-3}{\sqrt{x}-2}&=\lim_{x\ra4}\frac{\sqrt{1+2x}-3}{\sqrt{x}-2}\cdot\frac{\sqrt{1+2x}+3}{\sqrt{1+2x}+3}\\
        &=\lim_{x\ra4}\frac{1+2x-9}{(\sqrt{x}-2)(\sqrt{1+2x}+3)}\\
        &=\lim_{x\ra4}\frac{2(x-4)}{(\sqrt{x}-2)(\sqrt{1+2x}+3)}\\
        &=\lim_{x\ra4}\frac{2(x-4)}{(\sqrt{x}-2)(\sqrt{1+2x}+3)}\cdot\frac{\sqrt{x}+2}{\sqrt{x}+2}\\
        &=\lim_{x\ra4}\frac{2(x-4)(\sqrt{x}+2)}{(x-4)(\sqrt{1+2x}+3)}\\
        &=\lim_{x\ra4}\frac{2(\sqrt{x}+2)}{\sqrt{1+2x}+3}\\
        &=\frac{2(\sqrt{4}+2)}{\sqrt{1+2\cdot4}+3}\\
        &=\frac43
    \end{align*}
\end{proof}


\question[4] \begin{claim}
    \[\lim_{x\ra0}\frac{x^2}{\sqrt{1+5x}-(1+x)}=0\]
\end{claim}

\begin{proof}
    \begin{align*}
        \lim_{x\ra0}\frac{x^2}{\sqrt{1+5x}-(1+x)}&=\lim_{x\ra0}\frac{x^2}{\sqrt{1+5x}-(1+x)}\cdot\frac{\sqrt{1+5x}+(1+x)}{\sqrt{1+5x}+(1+x)}\\
        &=\lim_{x\ra0}\frac{x^2(\sqrt{1+5x}+(1+x))}{(1+5x)-(1+x)^2}\\
        &=\lim_{x\ra0}\frac{x^2(\sqrt{1+5x}+(1+x))}{1+5x-1-2x-x^2}\\
        &=\lim_{x\ra0}\frac{x^2(\sqrt{1+5x}+(1+x))}{-x^2+3x}\\
        &=\lim_{x\ra0}\frac{x(\sqrt{1+5x}+(1+x))}{-x+3}\\
        &=0
    \end{align*}
\end{proof}

\newpage
\section*{Exercise 2}

\begin{definition}
    Let $f(x)$ be defined on $(a,+\infty)$ for some $a\in\R$. We say that $f(x)$ tends to $k\in\R$ when $x$ tends to $+\infty$, and denote it by \[\lim_{x\ra+\infty}f(x)=k,\] if \(\forall\ep>0, \exists A\in\R, \text{ s.t. } \forall x > A, |f(x)-k|<\ep.\)
\end{definition}

In this exercise, we study some limits of average of functions, that are of similar spirits to those of the sequences we encountered before. To this end, assume that 
\begin{enumerate}
    \item $f(x)$ is defined on some interval $(a,+\infty)$, where $a\in\R$ and 
    \item $\forall b>a, f(x)$ is bounded on $(a,b)$ (be careful: this does not mean that $f(x)$ is bounded on $a,+\infty$)).
\end{enumerate}

\question[1] Prove that if $\displaystyle\lim_{x\ra+\infty}(f(x+1)-f(x))=k$, then \[\displaystyle\lim_{x\ra+\infty}\frac{f(x)}{x}=k.\]

\begin{proof}
    
\end{proof}

\question[2] Prove that if $\forall x\in (a,+\infty), f(x)\geq C > 0$ for some (fixed) positive $C$, and $\displaystyle\lim_{x\ra+\infty}\frac{f(x+1)}{f(x)}=k$, then \[\lim_{x\ra+\infty}(f(x))^{\frac1x}=k.\]


\newpage
\section*{Exercise 3}

Consider the following Riemann function $f:\R\to\R$ defined

\[f(x)=\begin{cases}
    \frac1n, &x=\frac{m}n\in\Q\setminus\{0\}, \gcd(m,n)=1, n\in\N; \\ 0,& x\in\Q^c\cup\{0\}.
\end{cases}\]

\question[1] Prove that $f(x)$ is not continuous at any $x\in\Q\setminus\{0\}$.

\begin{proof}
    We want to show that $\forall a\in\Q\setminus\{0\},\exists\ep>0,\forall\delta>0,\exists x\in\R, |x-a|<\delta\land|f(x)-f(a)|\geq\ep$. Fix $a=\frac{m}{n}\in\Q\setminus\{0\}$ where $\gcd(m,n)=1$, choose $\ep=\frac{f(a)}{2}=\frac1{2n}$, then $\forall\delta>0$, choose $x=\frac{m}{n}+\frac{\delta}{2}$ if $\delta$ is irrational otherwise $x=\frac{m}{n}+\frac{\delta}{\sqrt2}$ (to ensure $x$ is irrational). Then if $\delta$ is irrational, $|x-a|=\frac{\delta}{2}<\delta$ and $|f(x)-f(a)|=|0-\frac1n|=\frac1n\geq\frac1{2n}=\ep$; if $\delta$ is rational, $|x-a|=\frac{\delta}{\sqrt2}<\delta$ and $|f(x)-f(a)|=|0-\frac1n|=\frac1n\geq\frac1{2n}=\ep$. Thus by definition $f(x)$ is not continuous at all $a\in\Q\setminus\{0\}$.
\end{proof}

\question[2] Prove that $f(x)$ has left and right limit at any $x\in\Q$, i.e., \[\forall x\in\Q, \displaystyle\lim_{y\ra x^+}f(y), \lim_{y\ra x^-}f(y)\text{ exist.}\]

\begin{proof}
    It suffices to show the limit exists, thus by HW8 we know this implies the existence of left and right limits. I claim that the limit of any $a\in\Q$ is $0$. Fix $a\in\Q$, fix $\ep>0$, then by Archimedean Property choose some $N\in\N$ s.t. $\frac1N<\ep$. 

    Choose $\delta=\min(\{|\frac{p}{q}-a|:p\in\Z\cap[-(N-1), (N-1)], q\in\N, q\leq N\})$ (finite and non-empty thus is well-defined), then consider $x\in\R$ s.t. $0<|x-a|<\delta:$ if $x$ is irrational then we immediately get $|f(x)-0|=0<\ep$; if $x$ is rational then $x$ cannot be any of $\pm\frac11; \pm\frac12; \pm\frac13, \pm\frac23; \ldots; \pm\frac1N, \ldots, \pm\frac{N-1}{N}$ by our choose of $\delta$, so, $|f(x)-0|\leq\frac1{N}<\ep$.

    Hence, as $\displaystyle\lim_{y\ra x}f(y)=0$ for all $x\in\Q$, this implies the existence of left and right limits, completing our proof.

\end{proof}

\question[3] Prove that $f(x)$ is continuous for all $x\in\Q^c\cup\{0\}$. Moreover, conclude the type of discontinuity for $x\in\Q\setminus\{0\}$.

\begin{proof}
    (Similar to E3Q2). We want to show that $\forall a\in\Q^c\cup\{0\},\forall\ep>0,\exists\delta>0,\forall x\in\R, |x-a|<\delta\Ra|f(x)-f(a)|<\ep$. Fix $a\in\Q^c\cup\{0\}$, fix $\ep>0$, then by Archimedean Property choose some $N\in\N$ s.t. $\frac1N<\ep$.

    Choose $\delta=\min(\{|\frac{p}{q}-a|:p\in\Z\cap[-(N-1), (N-1)], q\in\N, q\leq N\})$ (finite and non-empty thus is well-defined), then consider $x\in\R$ s.t. $0<|x-a|<\delta:$ if $x$ is irrational then we immediately get $|f(x)-f(a)|=|0-0|=0<\ep$; if $x$ is rational then $x$ cannot be any of $\pm\frac11; \pm\frac12; \pm\frac13, \pm\frac23; \ldots; \pm\frac1N, \ldots, \pm\frac{N-1}{N}$ by our choose of $\delta$, so, $|f(x)-f(a)|=|f(x)-0|=f(x)\leq\frac1{N}<\ep$.

    Thus, by definition $f(x)$ is continuous for all $x\in\Q^c\cup\{0\}$. Moreover, since $f(x)$ is not continuous at any $x\in\Q\setminus\{0\}$, and we have shown that $f(x)$ has left and right limits at any $x\in\Q$ (moreover they are equal), thus $f(x)$ is removable discontinuous at all $x\in\Q\setminus\{0\}$, showing it is the first type of discontinuity.
\end{proof}

\question[4] Name two functions $f_1(x), f_2(x):\R\to\R$, satisfying the following conditions, respectively:
\begin{itemize}
    \item $f_1(x)$ is not continuous at uncountably many points, moreover all discontinuous points are of the second type of discontinuity.
    \item $f_2(x)$ is not continuous at countably many points, moreover all discontinuous points are of the first type of discontinuity.
\end{itemize}

Hint: Name two great German mathematicians. 

\begin{claim}
    $f_1(x)$ can be the Dirichlet function: \[
    f_1(x)=\begin{cases}
        1 & x\in\Q \\ 
        0 & x\notin\Q
    \end{cases}
    \]
    $f_2(x)$ is just the Thomae's function defined for this exercise: \[f_2(x)=\begin{cases}
    \frac1n, &x=\frac{m}n\in\Q\setminus\{0\}, \gcd(m,n)=1, n\in\N; \\ 0,& x\in\Q^c\cup\{0\}.
\end{cases}\]

\end{claim}
\begin{proof}
    For $f_1(x)$ it is sufficient to show for all $x\in\R$  the left limit does not exist, which implies the function is nowhere continuous and is essentially discontinuous everywhere (uncountably many points).  

    Suppose ad absurdum left limit exists at point $a\in\R$, namely $\exists L\in\R, \forall\ep>0, \exists\delta>0, x\in(a-\delta,a)\implies |f(x)-L|<\ep$. However by the density of irrationals and rationals, we can choose $x_1\in\R\setminus\Q$ and $x_2\in\Q$ s.t. $x_1\in(a-\delta,a)\land |f(x_1)-L|=|L|<\ep$ and $x_2\in(a-\delta,a)\land|f(x_2)-L|=|1-L|<\ep$. However $|L|<\ep$ and $|1-L|<\ep$ imply $-\ep<\frac12<\ep$, contradicting the case when $\ep=\frac14$, thus left limit does not exist for all $x\in\R$. 

    Hence by HW8 we have the limit does not exist at all $x\in\R$ and so the function $f_1$ is continuous nowhere (uncountably many), moreover by definition they are the second type of discontinuity. 
    
    % Suppose ad absurdum limit exists at point $a\in\R$, that is, $\forall\ep>0, \exists\delta>0$ s.t. $0<|x-a|<\delta\implies|f(x)-L|<\ep$ for some $L\in\R$. However, by the density of irrationals and rationals, we can choose $x_1\in\R\setminus\Q$ and $x_2\in\Q$ such that
    % $0<|x_1-a|<\delta\land|f(x_1)-L|=|L|<\ep$ and $0<|x_2-a|<\delta\land |f(x_2)-L|=|1-L|<\ep$, however $|L|<\ep$ and $|1-L|<\ep$ implies $-\ep<\frac12<\ep$, contradicting the case when, e.g., $\ep=\frac14$, thus contradiction, showing there is no such point $a\in\R$ s.t. limit of $f_1$ exists at $a$.  
\end{proof}
\begin{proof}
    For $f_2(x)$, from E3Q1, Q2, Q3 it is only not continuous for $x\in\Q\setminus\{0\}$ (which is countably many), moreover as their limits exist, they are the first type of discontinuity (removable). 
\end{proof}

\end{document}
\documentclass{homework}
\author{Joseph Siu}
\class{MAT157: Analysis I}
\date{\today}
\title{Homework 1}
%\address{Joseph Siu}

\usepackage{color}
\theoremstyle{remark}
\newtheorem*{claim}{Claim}

% Symbols
\newcommand*{\eg}{\leavevmode\unskip , e. g., \ignorespaces}
\newcommand*{\ie}{\leavevmode\unskip, i. e., \ignorespaces}
\newcommand{\nil}{\varnothing}
\AtBeginDocument{\def\O{\cal{O}}} % Big Oh
\AtBeginDocument{\def\C{\bb{C}}} % Complex
\newcommand{\R}{\bb{R}} % Reals
\newcommand{\Q}{\bb{Q}} % Rationals
\newcommand{\Z}{\bb{Z}} % Integers
\newcommand{\N}{\bb{N}} % Naturals
\renewcommand{\P}{\bb{P}} % Primes
\newcommand{\F}{\bb{F}} % Field
\newcommand{\GF}[1][2]{\bb{F}_{#1}} % Galois Field
\newcommand{\modulo}[1][n]{\Z/#1\Z} % Modulo class n
\newcommand{\ra}{\rightarrow}
\newcommand{\Ra}{\Rightarrow}
\newcommand{\?}{\stackrel{?}{=}}
\newcommand{\is}{\equiv}
\newcommand{\al}{\alpha}
\newcommand{\ep}{\varepsilon}
\renewcommand{\phi}{\varphi}
\newcommand{\p}{\partial}
\newcommand{\injective}{\hookrightarrow}
\newcommand{\surjective}{\twoheadrightarrow}
\newcommand{\bijective}{\hookrightarrow\mathrel{\mspace{-15mu}}\rightarrow}
\newcommand{\derivative}[2][x]{\frac{\D #2}{\D #1}}
\newcommand{\ceil}[1]{\left\lceil#1\right\rceil}
\newcommand{\floor}[1]{\left\lfloor#1\right\rfloor}
\newcommand{\near}[1]{\left\lfloor#1\right\rceil}
\newcommand{\arr}[1]{\left\langle#1\right\rangle}
\newcommand{\paren}[1]{\left(#1\right)}
\newcommand{\brk}[1]{\left[#1\right]}
\newcommand{\abs}[1]{\left|#1\right|}
\newcommand{\curl}[1]{\left\{#1\right\}}

\begin{document} \maketitle


\question A tautology is αmathematical statement that is always true. A contradiction is a mathematical statement that is always false.\\\\Let $P$ be a any mathematical statement. Determine if each of the following statements is a tautology, or a contradiction, or neither.\\

\begin{claim}
(a)   $P\wedge\neg P$ is a contradiction.
\end{claim}

\begin{center}
    \begin{tabular}{@{ }c | c@{ }@{ }c@{ }@{ }c@{ }@{ }c@{ }@{ }c@{ }@{ }c}
P &  & P & $\land$ & $\lnot$ & P & \\
\hline 
T &  & T & \textcolor{red}{F} & F & T & \\
F &  & F & \textcolor{red}{F} & T & F & \\
\end{tabular}
\end{center}
\begin{proof}
    There are 2 cases for this statement. First case when $P$ is true, this implies $\neg P$ is false, and so $P\land\neg P$ must also be false since $\neg P$ is false. Second case when $P$ is false, this immediately shows $P\land\neg P$ is false as $P$ is false. Therefore, the statement $P\land\neg P$ is false in all cases (always false), by the definition, this is a contradiction.
\end{proof}


\begin{claim}
(b) $P\lor\lnot P$ is a tautology.
\end{claim}

\begin{center}
    %NOTE: requires \usepackage{color}
\begin{tabular}{@{ }c | c@{ }@{ }c@{ }@{ }c@{ }@{ }c@{ }@{ }c@{ }@{ }c}
P &  & P & $\lor$ & $\lnot$ & P & \\
\hline 
T &  & T & \textcolor{red}{T} & F & T & \\
F &  & F & \textcolor{red}{T} & T & F & \\
\end{tabular}
\end{center}
\begin{proof}
    There are 2 cases for this question. First case when $P$ is true, this immediately shows $P\lor\neg P$ is true as $P$ is true. Second case when $P$ is false, this implies $\neg P$ is true, and for this case $P\lor\neg P$ is also true as $\neg P$ is true. Therefore, the statement $P\lor\neg P$ is true in all cases (always true), by the definition, this is a tautology. 
\end{proof}

\begin{claim}
    (c) $\begin{aligned}P\Rightarrow\neg P\end{aligned}$ is neither a tautology nor contradiction. 
\end{claim}

\begin{center}
    %NOTE: requires \usepackage{color}
\begin{tabular}{@{ }c | c@{ }@{ }c@{ }@{ }c@{ }@{ }c@{ }@{ }c@{ }@{ }c}
P &  & P & $\rightarrow$ & $\lnot$ & P & \\
\hline 
T &  & T & \textcolor{red}{F} & F & T & \\
F &  & F & \textcolor{red}{T} & T & F & \\
\end{tabular}
\end{center}
\begin{proof}
    There are 2 cases for this question.  First case when $P$ is true, we know $\neg P$ is false, and so the implication $P\implies\neg P$ is false. Second case when $P$ is false the implication $P\implies\neg P$ is immediately true (by the definition of implication). Therefore, as the statement is neither always true nor always false, this is also neither a tautology nor contradiction.
\end{proof}

\newpage

\question Suppose we have a group of 10 people and the friendship is mutual (i.e., if A considers B as a friend if and only if B considers A as a friend).
Show that there exist two people who have the same number of friends.

\begin{claim}
    If there is a group of 10 people and the friendship is mutual, then there exist two people who have the same number of friends. 
\end{claim}

\begin{proof}
    Suppose we have a group of 10 people, denote the number of friends of a person by $n$, which $n\in[0,9]\cap\mathbb{Z}$. We notice that $n$ only has 10 possibilities: $n=0,1,\cdots,8,9$. 

    Here is the proof of the non-existence of 10 people having 10 distinct number of friends ($n)$:

    Let 10 people be in order from 1 to 10. The total number of friends of all 10 people can be expressed as $\Sigma_{i=1}^{10}n_i$ which $n_i$ is the number of friends of the $i$th person. We know that if 2 persons becomes friends, then the total number of friends adds 2, and their $n$ values add 1 respectively. Therefore, the total number of friends must be an even number as there cannot be a relation which one person is another person's friend but not the other way around (always in 2 directions). But in this situation, when 10 people have distinct $n$ values, which are $\{0,1,2,3,4,5,6,7,8,9\}$, the value $\Sigma_{i=1}^{10}n_i=0+1+2+\ldots+8+9=45$ is not an even number, contradicting the fact of the total number of friends being even. So, this case does not exist. 

    We have proven that there can never be a situation which 10 people (within the group) all have distinct $n$, then there can at most only occur 9 distinct $n$ values among the 10 people, by the Pigeonhole principle, define 10 groups (with 10 distinct $n$ values) which are classified based on the person's $n$ value, divide 10 people into 9 groups (there must exists an empty group otherwise contradicting the fact that the 10 people cannot have 10 distinct $n$ values, and so we eliminate the empty group), there is at least one group consists of no less than $\frac{10}9$ people, because we define the number of humans in non-negative number, and so this is equivalent to there must exists at least 2 persons such that they have the same number of friends, this completes our proof. 
\end{proof}

\newpage

\question Four professional heroes, Saitama, Genos, Flash, and King have participated in a tournament, following a single round-robin schedule(i.e., each participant plays every other participant once). It is known that

(a) Saitama, Genos, and Flash have won the same number of games.

(b) Saitama has beaten King.

\begin{claim}
    King has not won any game (has won 0 game). 
\end{claim}

\begin{proof}
    We know that each participant plays every other participant once, so the total number of games is 3+2+1=6. We also know that Saitama, Genos, and Flash have the same number of games won, denote the number as $n$ which $n\in\mathbb{N}\cup\{0\}$. Since Saitama has beaten King, this implies $n\geq1$. Denote the number of games King won as $k$ which $k\in\mathbb{N}\cup\{0\}$, the total number of games (won) is equal to the sum of all players' number of games won, $6=n+n+n+k=3n+k$. Because $n$ and $k$ are both non-negative integers, so $3n+k=6 \iff k=6-3n$, only $n=0,1,2$ imply $k$ is non-negative, but also $n\geq1$, the only possible values of $n$ are 1 and 2. 

    Consider the case when $n = 1$, $k=6-3n=6-3(1)=3$, this implies King has own all of his/her games (every player only has 3 games), contradicting the given fact that Saitama has beaten King (King must has won at most 2 times ). Thus, this case is impossible to occur.

    Therefore, it is only possible that $n = 2$, the equivalent meaning is, Saitama, Genos, and Flash have all won 2 games, King has not won any game ($k=6-3(2)=0$).  In other words, King has won 0 game.
\end{proof}

\newpage
\question (skipping this question) Read the following two proofs and determine which part goes wrong:\\

(a) "1 is the greatest natural number."

\begin{proof}
    By contradiction Suppose otherwise that some $a\in\mathbb{N}$ , rather than 1, is the greatest natural number. Then $a> $1. It follows that $a^2=a\cdot a>1\cdot a=a$. However, $a^2>a$, contradiction. We conclude that l is the greatest natural number.
\end{proof}
The problem in this proof is, $a^2=a\cdot a > 1\cdot a = a$, is not contradicting $a^2>a$. We start by $a>1,$ multiplying both sides by $a$ ($a$ is a natural number), we get $a\cdot a>1\cdot a$, which is equivalent to $a^2>a$, following the fact that $a^2>a$, not a contradiction. The writer probably interpreted the statement wrong, they grouped $a^2=(a\cdot a > 1\cdot a) = a$. 

(b) "..."



\newpage
\question \textbf{Definition 2} (Divisibility). \textit{Given two integers $p,q\in\mathbb{Z},q\neq 0,$ we say that $q$ \textbf{divides} $p$ (or equivalently that $p$ is divisible by $q$), denoted by $q\mid p$, if there exists $k\in\mathbb{Z}$ such that $p=kq.$ We call $q$ a divisor of $p$, and $p$ a multiple of q.}

Determine if each of the following statement is correct or not, by proving it or giving a counter example. 

\begin{claim}
    (a) $\begin{aligned}\forall x\in\mathbb{Z},\exists y\in\mathbb{Z}\end{aligned}$ s.t. $(3|x+y)\lor(5|x+y)$ is a true (correct) statement. 
\end{claim}

\begin{proof}
    For any $x\in\mathbb{Z},$ pick $y=14x\in\mathbb{Z},$ then $x+y=15x \iff x+y=3\cdot5\cdot x.$ That is, since $x+y=3(5x)$ implies $3\mid x+y$ and $x+y=5(3x)$ implies $5\mid x+y.$ We can conclude $\begin{aligned}\forall x\in\mathbb{Z},\exists y\in\mathbb{Z}\end{aligned}$ s.t. $(3|x+y)\lor(5|x+y)$ is always correct. 
\end{proof}

\begin{claim}
    (b) $\exists y\in\mathbb{Z}$ s.t. $\forall x \in \mathbb{Z},(3|x+y)\lor(5|x+y)$ is a false (incorrect) statement. 
\end{claim}

\begin{proof}
    We show this statement is false by showing the negation of it is true, i.e., we prove that $\forall y \in\mathbb{Z}$ s.t. $\exists x\in\mathbb{Z}, (3\nmid x+y)\land(5\nmid x+y)$.

    We let $x=7-y,$ i.e., $x+y=7-y+y=7,$ and we know that $3\nmid7\land5\nmid7$ is true since 3 and 7 are co-prime, and 5 and 7 are co-prime, this means 7 cannot be expressed as a multiple of 3 and 5 respectively. 
    
    Therefore, we have shown that the negation of the statement  $\forall y \in\mathbb{Z}$ s.t. $\exists x\in\mathbb{Z}, (3\nmid x+y)\land(5\nmid x+y)$ is true, implying the statement $\exists y\in\mathbb{Z}$ s.t. $\forall x \in \mathbb{Z},(3|x+y)\lor(5|x+y)$ is false, completing our proof.  
\end{proof}


Here is a longer approach. 

\begin{proof}
    We prove this statement is false by showing the negation of it is true, i.e., we prove that $\forall y \in\mathbb{Z}$ s.t. $\exists x\in\mathbb{Z}, (3\nmid x+y)\land(5\nmid x+y)$
    
    We split the cases of $y$
    
    Case 1, when $3\mid y \land 5\mid y$. we let $x$ be a number that is not divisible by both 3 and 5, $x=2$. Then $3\mid y\implies \exists a\in\mathbb{Z},y=3a$ and $5\mid y\implies \exists b\in\mathbb{Z},y=5b.$ This shows $x+y=3a+2=5b+2$. We notice $3\nmid 3a+2$ and $5\nmid 5b+2$ since we cannot express the number 2 in terms of 3 or 5. Thus, we have shown that the negation of the statement is true for this case
    
    Case 2. If  $3\mid y\land 5\nmid y$, We make $y$ becomes divisible by 5 by assigning $x=4y+2$, which then $x+y=5y+2,$ and we know $3\mid y\implies \exists a\in\mathbb{Z},y=3a,$ so $x+y=5\cdot3\cdot a+2$. Showing both $3\nmid 3(5a)+2$ and $5\nmid 5(3a)+2$, this case is verified as true
    
    Case 3. If $3\nmid y \land 5\mid y.$ We follow the similar steps, let $x=2y+2$ so that $x+y=3y+2$. Because $5\mid y\implies \exists b\in\mathbb{Z},y=5b,$ so $x+y=3\cdot5\cdot b+2$. Similarly we have shown the negation of the statement statement holds true for this case
    
    Case 4. If $3\nmid y \land 5\nmid y$. We set $x=0$ and the proof is immediately completed as $x+y=0+y=y$ and we already know $3\nmid y \land 5\nmid y$.
    
    For all cases of $y$ we can always find a $x$ value such that the negation of the statement holds true, this implies the original statement $\exists y\in\mathbb{Z}$ s.t. $\forall x ∈ Z,(3|x+y)\lor(5|x+y)$ is false, completing our proof.
 \end{proof}



\end{document}
